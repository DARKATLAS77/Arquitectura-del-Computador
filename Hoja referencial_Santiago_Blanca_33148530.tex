%------------------------
% Latex CV Template for COS Northeastern University Faculty
% Author: Zoe Kearney

% Formatting: https://www.overleaf.com/latex/templates/coles-resume-template/qhpynjcvjpcj

% Follows: Template Dossier Documents, Office of the Provost, Northeastern; July 1, 2025 version (https://provost.northeastern.edu/faculty-affairs/). 

%Disclaimer: This template is intended to complement Northeastern University’s dossier guidelines for tenure and promotion cases, as well as discipline-specific CV standards.

% License: LaTeX Project Public License 1.3c
%------------------------

% Document class and font size
\documentclass[12pt]{article}

% Packages
\usepackage[utf8]{inputenc} % For input encoding
\usepackage{geometry} % For page margins
\geometry{a4paper, margin=1in} % Set paper size and margins
\usepackage{titlesec} % For section title formatting
\usepackage{enumitem} % For itemized list formatting
\usepackage{hyperref} % For hyperlinks
\usepackage{lipsum} % For filler text
\usepackage{changepage} %For Indenting

% Formatting
\setlist{noitemsep} % Removes item separation
\titleformat{\section}{\large\bfseries}{\thesection}{1em}{}[\titlerule] % Section title format
\titlespacing*{\section}{0pt}{\baselineskip}{\baselineskip} % Section title spacing
\newenvironment{subs} %Indenting
  {\adjustwidth{2em}{0pt}}
  {\endadjustwidth}

%-------------------------------------------------------------------------------
%-------------------------------------------------------------------------------
% Begin document
\begin{document}

% Disable page numbers
\pagestyle{empty}

% Header
\begin{center}
    \textbf{\Large Hoja Referencial Mips 32 }\\[12pt] % Name
    \textbf{Santiago Blanca 33148530}\\[1pt] % CV
     
\end{center}


%------------------------

    \begin{center}
 {\large \textbf{Categorias}}
    \noindent
    \end{center}
%------------------------------------
\section*{Aritmética}

\begin{center}
\begin{tabular}{l l l}
    \textbf{Instrucción} & \textbf{Ejemplo} & \textbf{Significado} \\
    \hline
    add & \texttt{add \$s1,\$s2,\$s3} & \$s1 = \$s2 + \$s3 \\
    sub & \texttt{sub \$s1,\$s2,\$s3} & \$s1 = \$s2 - \$s3 \\
    addi & \texttt{addi \$s1,\$s2,100} & \$s1 = \$s2 + 100 \\
\end{tabular}
\end{center}


%------------------------------------
\section*{Transferencia de datos}

\begin{center}
\begin{tabular}{l l l}
    \textbf{Instrucción} & \textbf{Ejemplo} & \textbf{Significado} \\
    \hline
    lw  & \texttt{lw \$s1,100(\$s2)}  & \$s1 = Memory[\$s2 + 100] \\
    sw  & \texttt{sw \$s1,100(\$s2)}  & Memory[\$s2 + 100] = \$s1 \\
    lb  & \texttt{lb \$s1,100(\$s2)}  & \$s1 = Memory[\$s2 + 100] \\
    sb  & \texttt{sb \$s1,100(\$s2)}  & Memory[\$s2 + 100] = \$s1 \\
\end{tabular}
\end{center}

%------------------------------------  
\section*{Lógica}

\begin{center}
\begin{tabular}{l l l}
    \textbf{Instrucción} & \textbf{Ejemplo} & \textbf{Significado} \\
    \hline
    and  & \texttt{and \$s1,\$s2,\$s3}  & \$s1 = \$s2 \& \$s3 \\
    or   & \texttt{or \$s1,\$s2,\$s3}   & \$s1 = \$s2 \;|\; \$s3 \\
    andi & \texttt{andi \$s1,\$s2,100}  & \$s1 = \$s2 \& 100 \\
    sll  & \texttt{sll \$s1,\$s2,10}    & \$s1 = \$s2 \texttt{<<} 10 \\
    srl  & \texttt{srl \$s1,\$s2,10}    & \$s1 = \$s2 \texttt{>>} 10 \\
\end{tabular}
\end{center}


%------------------------------------
\section*{Salto condicional}

\begin{center}
\begin{tabular}{l l l}
    \textbf{Instrucción} & \textbf{Ejemplo} & \textbf{Significado} \\
    \hline
    beq & \texttt{beq \$s1,\$s2,L}  & si (\$s1 == \$s2) ir a L \\
    bne & \texttt{bne \$s1,\$s2,L}  & si (\$s1 != \$s2) ir a L \\
    slt & \texttt{slt \$s1,\$s2,\$s3} &
        \begin{tabular}[c]{@{}l@{}}si (\$s2 < \$s3) \$s1 = 1 \\ si no, \$s1 = 0\end{tabular} \\
    slti & \texttt{slti \$s1,\$s2,100} &
        \begin{tabular}[c]{@{}l@{}}si (\$s2 < 100) \$s1 = 1 \\ si no, \$s1 = 0\end{tabular} \\
\end{tabular}
\end{center}

%------------------------------------
\section*{Salto incondicional}

\begin{center}
\begin{tabular}{l l l}
    \textbf{Instrucción} & \textbf{Ejemplo} & \textbf{Significado} \\
    \hline
    j   & \texttt{j 2500} & ir a 10000 \\
    jr  & \texttt{jr \$ra} & ir a \$ra \\
    jal & \texttt{jal 2500} & \$ra = PC + 4; ir a 10000 \\
\end{tabular}
\end{center}

%------------------------

% End document
\end{document}
%-------------------------------------------------------------------------------
%-------------------------------------------------------------------------------